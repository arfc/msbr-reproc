\section{Methods}

Liquid-fueled system modeling with contemporary reactor physics codes is difficult because most of these codes were developed for analyzing solid-fueled reactors. In fact, world-wide commercial reactors fleet today is represented by systems with solid fuel. Liquid-fueled systems ability to continuously remove fission products and add fissile and/or fertile elements is the main challenge for depletion simulations. SaltProc takes into account online separations and feeds using SERPENT 2 continuous-energy Monte Carlo neutron transport and depletion code.

\subsection{Molten Salt Breeder Reactor design and model description}
The \gls{MSBR} vessel has a diameter of 680 cm and a height of 610 cm. It contains a molten fluoride fuel-salt mixture that generates heat in the active core region and transports that heat to the primary heat exchanger by way of the primary salt pump. In the active core region, the salt flows through channels in moderating and reflecting graphite blocks. Salt at about 565$^{\circ}$C enters the central manifold at the bottom via four 40.64-cm-diameter nozzles and flows upward through channels in the lower plenum graphite. The fuel salt exits at the top at about 704$^{\circ}$C through four equally spaced nozzles which connect to the salt-suction pipes leading to primary circulation pumps. The fuel salt drain lines connect to the bottom of the reactor vessel inlet manifold.

Figure~\ref{fig:serpent_plan_view} demonstrate the configuration of the \gls{MSBR} vessel, core configuration, ``fission" (zone I) and ``breeding" (zone II) regions inside the vessel. The core has two radial zones bounded by a solid cylindrical graphite reflector and the vessel wall. The central zone, zone I, in which 13\% of the volume is fuel salt and 87\% graphite. Zone I composed of 1,320 graphite cells, 2 graphite control rods, and 2 safety\footnote{These rods needed for emergency shutdown only.} rods. The under-moderated zone, zone II, with 37\% fuel salt, and radial reflector, surrounds the zone I core region and serves to diminish neutron leakage. Zones I and II are surrounded radially and axially by fuel salt (figure~\ref{fig:serpent_zoneII}). This space for fuel is necessary for injection and flow of molten salt.

Since reactor graphite experiences significant dimensional changes due to neutron irradiation, the reactor core was designed for periodic replacement. Based on the irradiation experimental data from \gls{MSRE}, core graphite lifetime is about 4 years and reflector graphite lifetime is 30 years \cite{robertson_conceptual_1971}.

\begin{figure}[hbp!] % replace 't' with 'b' to 
  \centering
  \includegraphics[width=\textwidth]{view_serpent.png}
  \caption{Plan and elevation view of SERPENT 2 \gls{MSBR} model developed in this work.}
  \label{fig:serpent_plan_view}
\end{figure}

Moreover, it was decided to remove and install the core graphite as an assembly rather than by individual blocks, because it is relatively easier for maintenance personnel and has lower probability of radioactive elements escape due to used blocks damage during removal. In addition, handling the core as an assembly also allows the replacement core to be carefully preassembled and tested under factory conditions.

There are eight symmetric graphite slabs with a width of 15.24 cm in zone II, one of which is illustrated in Figure~\ref{fig:serpent_zoneII}. The holes in the centers are for the core lifting rods used during the core replacement operations. These holes also allow a portion of the fuel salt to flow to the top of the vessel for cooling the top head and axial reflector. Figure~\ref{fig:serpent_zoneII} also demonstrates the 5.08-cm-wide annular space between the removable core graphite in zone II-B and the permanently mounted reflector graphite. This annulus consists entirely of fuel salt, provides space for moving the core assembly, helps compensate the elliptical dimensions of the reactor vessel, and serves to reduce the damaging flux at the surface of the graphite reflector blocks. In this work, all figures of the core were generated using the built-in SERPENT 2 plotter. All calculations presented in this study were performed using SERPENT 2 version 2.1.30 on Blue Waters’ XE6 nodes. 

\begin{figure}[t!] % replace 't' with 'b' to 
  \centering
  \includegraphics[width=\textwidth]{ser_zone_II.png}
  \caption{Detailed view of \gls{MSBR} zone II model.}
  \label{fig:serpent_zoneII}
\end{figure}

\subsubsection{Core zone I}
The central region of the core, called zone I, is made up of graphite elements, each $10.16$cm$\times$10.16cm$\times$396.24cm. Zone I has 4 channels for control rods: two for graphite rods which both regulate and shim during normal operation, and two for backup safety rods consisting of boron carbide clad to assure sufficient negative reactivity for emergency situations.

These graphite elements have a mostly rectangular shape with lengthwise ridges at each corner that leave space for salt flow elements. Various element sizes reduce the peak damage flux and power density in the center of the core to prevent local graphite damage. Zone I is well-moderated to achieve the desired fission power density. Figure~\ref{fig:I_element_ref} demonstrates the elevation and plan views of graphite elements of zone I \cite{robertson_conceptual_1971} and their SERPENT model \cite{rykhlevskii_full-core_2017}.

\begin{figure}[ht!] % replace 't' with 'b' to 
  \centering
  \includegraphics[width=\textwidth]{zone_I_element_ref.png}
  \caption{Graphite moderator elements for zone I \cite{robertson_conceptual_1971,rykhlevskii_full-core_2017}.}
  \label{fig:I_element_ref}
\end{figure}

\subsubsection{Core zone II}
Zone II which is undermoderated, surrounds zone I. Combined with the bounding radial reflector, zone II serves to diminish neutron leakage. This zone is formed of two kinds of elements: large-diameter fuel channels (zone II-A) and radial graphite slats (zone II-B). 

Zone II has 37\% fuel salt by volume and each element has a fuel channel diameter of 6.604cm. The graphite elements for zone II-A are prismatic and have elliptical-shaped dowels running axially between the prisms and needed to isolate the fuel salt flow in zone I from that in zone II. Figure~\ref{fig:II_element_ref} shows shape and dimensions of these graphite elements and their SERPENT model. Zone II-B elements are rectangular slats spaced far enough apart to provide the 0.37 fuel salt volume fraction. The reactor zone II-B graphite 5.08cm-thick slats vary in the radial dimension (average width is 26.67cm) as shown in figure~\ref{fig:serpent_zoneII}. Zone II serves as a blanket to achieve the best performance: a high breeding ratio and a low fissile inventory. The neutron energy spectrum in zone II is made harder to enhance the rate of thorium resonance capture relative to the fission rate, thus limiting the neutron flux in the outer core zone and reducing the neutron leakage \cite{robertson_conceptual_1971}. 

The main challenge was to accurately represent zone II-B because it has irregular elements with sophisticated shapes. From the \gls{ORNL} report \cite{robertson_conceptual_1971}, the suggested design of zone II-B has 8 irregularly-shaped graphite elements every 45$^\circ$ as well as salt channels. These graphite elements were simplified into right-circular cylindrical shapes  with central channels. Figure~\ref{fig:serpent_zoneII} illustrates this core region in the SERPENT model. The volume of fuel salt in zone II was kept exactly 37\%, so that this simplification did not considerably change the core neutronics. This is the only simplification made to the \gls{MSBR} geometry in this work. 

\begin{figure}[ht!] % replace 't' with 'b' to 
  \centering
  \includegraphics[width=\textwidth]{zone_II_element_ref.png}
  \caption{Graphite moderator elements for zone II-A \cite{robertson_conceptual_1971,rykhlevskii_full-core_2017}.}
  \label{fig:II_element_ref}
\end{figure}

\subsubsection{Material composition and normalization parameters}
The fuel salt, the reactor graphite, and the modified Hastelloy-N\footnote{ Hastelloy-N is very common in reactors now but have been studied and developed at \gls{ORNL} in a program that started in 1950s.} are materials unique of the \gls{MSBR} and were created at \gls{ORNL}. The initial fuel salt used the same density (3.35 g/cm$^3$) and composition LiF-BeF$_2$-ThF$_4$-$^{233}$UF$_4$ (71.75-16-12-0.25 mole \%) as the \gls{MSBR} design \cite{robertson_conceptual_1971}. The lithium in the molten salt fuel is fully enriched in $^{7}$Li because $^{6}$Li is a very strong neutron poison and becomes tritium upon neutron capture. 

For cross section generation, JEFF-3.1.2 neutron library was employed \cite{oecd/nea_data_bank_jeff-3.1.2_2014}. The specific temperature was fixed for each material to correctly model the Doppler-broadening of resonance peaks when SERPENT generates the problem-dependent nuclear data library. The isotopic composition of each material at the initial state was described in detail in the MSBR conceptual design study \cite{robertson_conceptual_1971} and has been applied to SERPENT model without any modification. Table~\ref{tab:msbr_tab} is a summary of the major \gls{MSBR} parameters used by this model \cite{robertson_conceptual_1971}. 

%%%%%%%%%%%%%%%%%%%%%%%%%%%%%%%%%%%%%%%%
\begin{table}[h!]
        %\centering
        \caption{Summary of principal data for MSBR \cite{robertson_conceptual_1971}.}
        \begin{tabularx}{\textwidth}{ s  s}
        \hline
                Thermal capacity of reactor           		& 2250 MW(t)
                \\ 
                Net electrical output                 		& 1000 MW(e) 
                \\  
                Net thermal efficiency        				& 44.4\%
                \\  
                Salt volume fraction in central zone I		& 0.13
                \\ 
                Salt volume fraction in outer zone II       & 0.37
                \\ 
                Fuel salt inventory (Zone I)                & 8.2 m$^3$	
                \\ 
                Fuel salt inventory (Zone II)               & 10.8 m$^3$	
                \\ 
                Fuel salt inventory (annulus)               & 3.8 m$^3$	
                \\  
                Total fuel salt inventory                   & 48.7 m$^3$	
                \\ 
                Fissile mass in fuel salt                   & 1303.7 kg	
                \\ 
                Fuel salt components                  & 
                LiF-BeF$_2$-ThF$_4$-$^{233}$UF$_4$	
                \\  
                Fuel salt composition                 & 
                71.75-16-12-0.25 mole\%
                \\
                Fuel salt density                    & 
                3.35 g/cm$^3$
                \\ \hline
        \end{tabularx}
        \label{tab:msbr_tab}
\end{table}
%%%%%%%%%%%%%%%%%%%%%%%%%%%%%%%%%%%%%%%%%%%%%%%%

\subsection{Online reprocessing method}
Removing specific chemical elements from a molten salt is a complicated task that requires intelligent design (e.g., chemical separations equipment design, fuel salt flows to equipment) and has a considerable economic cost. All liquid-fueled \gls{MSR} designs involve varying levels of online fuel processing. Minimally, volatile gaseous fission products (e.g. Kr, Xe) escape from the fuel salt during routine reactor operation and must be captured. Additional systems might be used to enhance removal of those elements. Most designs also call for the removal of noble and rare earth metals from the core since these metals act as neutron poisons. Some designs suggest a more complex list of elements to process (figure ~\ref{fig:periodic_tab}), including the temporary removal of protactinium from the salt or other regulation of the actinide inventory in the fuel salt \cite{ahmad_neutronics_2015}.

In the single-fluid \gls{MSBR} considered in this work, thorium, uranium, protactinium, and fission products are all mixed together in a single fluoride salt (FLiBe). Separation of thorium from lanthanide (atomic numbers 57 through 71) fission products is rather challenging because of their chemical similarities. 

\begin{figure}[htp!] % replace 't' with 'b' to 
  \centering
  \includegraphics[width=\textwidth]{periodic_map.png}
  \caption{Processing options for \gls{MSR} fuels \cite{ahmad_neutronics_2015}.}
  \label{fig:periodic_tab}
\end{figure}

\subsubsection{Fuel material flows}
The $^{232}$Th in the fuel absorbs thermal neutrons and produces $^{233}$Pa which then decays into the fissile $^{233}$U. Furthermore, the \gls{MSBR} design requires online reprocessing to remove all poisons (e.g. $^{135}$Xe), noble metals, and gases (e.g. $^{75}$Se, $^{85}$Kr) every 20 seconds. Protactinium presents a challenge, since it has a large absorption cross section in the thermal energy spectrum. Accordingly, $^{233}$Pa is continuously removed from the fuel salt into a protactinium decay tank to allow $^{233}$Pa to decay to $^{233}$U without poisoning the reactor. The reactor reprocessing system is designed to separate $^{233}$Pa from the molten-salt fuel over 3 days, hold it while $^{233}$Pa decays into $^{233}$U, and return it back to the primary loop. This feature allows the reactor to avoid neutron losses to protactinium, keeps fission products to a very low level, and increases the efficiency of $^{233}$U breeding. Table~\ref{tab:reprocessing_list} summarizes full list of nuclides and the cycle times used for modeling salt treatment and separations \cite{robertson_conceptual_1971}. 

%%%%%%%%%%%%%%%%%%%%%%%%%%%%%%%%%%%%%%%%
\begin{table}[ht!]
        \centering
        \caption{The effective cycle times for protactinium and fission products removal (reproduced from \cite{robertson_conceptual_1971}).}
        \begin{tabularx}{\textwidth}{ x | s | x }
        \hline 
        Processing group & \qquad\qquad\qquad Nuclides & Cycle time (at full power) \\ \hline 
        Rare earths & Y, La, Ce, Pr, Nd, Pm, Sm, Gd & 50 days \\ 
        \qquad & Eu & 500 days \\ 
        Noble metals & Se, Nb, Mo, Tc, Ru, Rh, Pd, Ag, Sb, Te & 20 sec \\
        Seminoble metals & Zr, Cd, In, Sn & 200 days \\
        Gases & Kr, Xe & 20 sec \\ 
        Volatile fluorides & Br, I & 60 days \\
        Discard & Rb, Sr, Cs, Ba & 3435 days \\ 
        Salt discard & Th, Li, Be, F & 3435 days \\ 
        Protactinium & $^{233}$Pa & 3 days \\ 
        Higher nuclides & $^{237}$Np, $^{242}$Pu & 16 years \\ 
        \end{tabularx}
        %\footnotetext{Chemical processing plant and gas separation system removing chemical elements (not isotopes) only. Isotopes instead of elements listed because other isotopes are short-lived and might be ignored.}
        \label{tab:reprocessing_list}
\end{table}
Th removal rates vary among nuclides in this reactor concept which dictate the necessary resolution of depletion calculations. If the depletion time intervals are very short, an enormous number of depletion steps are required to obtain the equilibrium composition. On the other hand, if the depletion  calculation time interval is too long, the impact of short-lived fission products is not captured. To compromise, the time interval for depletion calculations in this model was selected as 3 days to correlate with the removal interval of $^{233}$Pa and $^{232}$Th was continuously added to maintain the initial mass fraction of $^{232}$Th.

\subsubsection{The SaltProc modeling and simulation code}