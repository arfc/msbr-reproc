\section{Results}
The SaltProc online reprocessing simulation code is demonstrated in four applications: (1) analyzing  \gls{MSBR} neutronics and fuel cycle to find the equilibrium core composition and core depletion, (2) studying operational and safety parameters evolution during \gls{MSBR} operation, (3) demonstrating that in a single-fluid two-region \gls{MSBR} conceptual design the undermoderated outer core zone II works as a virtual ``blanket", reduces neutron leakage and improves breeding ratio due to neutron energy spectral shift, and (4) determining the effect of fission product removal on the core neutronics.

The neutron population per cycle and the number of active/inactive cycles were chosen to obtain balance between reasonable uncertainty for a transport problem ($\leq$ 15 pcm\footnote{ 1 pcm = 10$^{-5}\Delta k_{eff}/k_{eff}$} for effective multiplication factor) and computational time. The \gls{MSBR} depletion and safety parameter computations were performed on 64 Blue Waters XK7 nodes (two AMD 6276 Interlagos CPU per node, 16 floating-point Bulldozer core units per node or 32 ``integer" cores per node, nominal clock speed is 2.45 GHz). The total computational time for achieving equilibrium composition was approximately 9,900 node-hours (18 core-years.)

\subsection{Effective multiplication factor}
Figure~\ref{fig:keff} demonstrates the effective multiplication factors obtained using SaltProc and SERPENT 2. The effective multiplication factors are calculated after removing fission products listed in Table~\ref{tab:reprocessing_list} and adding the fertile material at the end of ``cycle time"\footnote{ The \gls{MSBR} program defined a ``cycle time" as the amount of time required to remove 100\% of a target nuclide from a fuel salt.} which was fixed at 3 days for this work. The effective multiplication factor fluctuates significantly as a result of the batch-wise nature of this online reprocessing strategy. 

First, SERPENT calculates the effective multiplication factor for the beginning of cycle time (fresh fuel composition for the first step). Next, it computes the new fuel salt composition for the end of a 3-day depletion step. The corresponding effective multiplication factor is much smaller than the previous one. Finally, SERPENT calculates $k_{eff}$ for the depleted composition after applying feeds and removals, and this increases accordingly since major reactor poisons (e.g. Xe, Kr) are removed, while fresh fissile material ($^{233}$U) from the protactinium decay tank is added. 
\begin{figure}[ht!] % replace 't' with 'b' to 
  \centering
  \includegraphics[width=\textwidth]{keff.png}
  \caption{Effective multiplication factor dynamics for full-core \gls{MSBR} model for a 60-year reactor operation. The confidence interval $\sigma=\pm15pcm$ is shaded.}
  \label{fig:keff}
\end{figure}
Additionally, the presence of rubidium, strontium, cesium, and barium in the core are disadvantageous to reactor physics. In fact, removal of these elements every 3435 days causes the multiplication factor to jump by approximately 450 pcm, and limits using the batch approach for online reprocessing simulation. Overall, the effective multiplication factor gradually decreases from 1.075 to $k_{eff} \approx 1.02$ at equilibrium after approximately 6 years of irradiation. 

\subsection{Fuel salt composition dynamics}
The analysis of the fuel salt composition evolution provides more comprehensive information about the equilibrium state. Figure~\ref{fig:adens_eq} shows number density of major nuclides which have a strong influence on the reactor core physics. Concentration of $^{233}$U, $^{232}$Th, $^{233}$Pa, and $^{232}$Pa in fuel salt change insignificantly after approximately 2500 days of operation. Particularly, $^{233}$U number density fluctuates less than 0.8\% in the time interval from 16 to 20 years of operation, hence,a quasi-equilibrium state was achieved after 16 years of reactor operation.
\begin{figure}[ht!] % replace 't' with 'b' to 
  \centering
  \includegraphics[width=\textwidth]{major_isotopes_adens.png}
  \caption{Number density of major nuclides during 60 years of reactor operation.}
  \label{fig:adens_eq}
\end{figure}
In contrast, a wide variety of nuclides, including fissile isotopes (e.g. $^{235}$U) and non-fissile strong absorbers (e.g. $^{234}$U), keep accumulating in the core. Figure~\ref{fig:fissile_short} demonstrate production of fissile isotopes in the core. In the end of considered operational time the core contains significant $^{235}$U ($\approx10^{-5}$ atom/b-cm), $^{239}$Pu ($\approx5\times10^{-7}$ atom/b-cm), and $^{241}$Pu ($\approx 5\times10^{-7}$ atom/b-cm). Meanwhile, the equilibrium number density of the target fissile isotope $^{233}$U was approximately 7.97$\times10^{-5}$ atom/b-cm. Thus, production of new fissile materials in the core as well as $^{233}$U breeding make it possible to compensate for negative effects of strong absorber accumulation and keep the reactor critical.
\begin{figure}[htp!] % replace 't' with 'b' to 
  \centering
  \includegraphics[width=\textwidth]{fissile_short.png}
  \caption{Number density of fissile in epithermal spectrum nuclides accumulation during the reactor operation.}
  \label{fig:fissile_short}
\end{figure}

\subsection{Neutron spectrum}
Figure~\ref{fig:spectrum} shows the normalized neutron flux spectrum for the full-core \gls{MSBR} model in the energy range from $10^{-8}$ to $10$ MeV. The neutron energy spectrum at equilibrium is harder than at startup due to $^{238}$Pu, $^{239}$Pu, $^{240}$Pu, $^{241}$Pu, and $^{242}$Pu accumulation in the core during reactor operation. 
\begin{figure}[ht!] % replace 't' with 'b' to force it to 
  \centering
  \includegraphics[width=\textwidth]{spectrum.png} 
  \caption{Neutron flux energy spectrum normalized by unit lethargy for initial and equilibrium fuel salt composition.}
  \label{fig:spectrum}
\end{figure}
Figure~\ref{fig:spectrum_zones} shows that zone I produced more thermal neutrons than zone II, corresponding to a majority of fissions occuring in the central part of the core. In the undermoderated zone II, the neutron energy spectrum is harder which leads to more capture of neutrons by $^{232}$Th and helps a achieve relatively high breeding ratio. Moreover, the (n,$\gamma$) resonance energy range in $^{232}$Th is from 10$^{-4}$ to 10$^{-2}$ MeV. Therefore, the moderator-to-fuel ratio for zone II was chosen to shift the neutron energy spectrum in this range. Furthermore, in the central core region (zone I), the neutron energy spectrum shifts to a harder spectrum over 20 years of reactor operation. In contrast, in the outer core region (zone II) a similar spectral shift takes place at a reduced scale. This resuls is in a good agreement with original ORNL report \cite{robertson_conceptual_1971} and most recent whole-core steady-state study \cite{park_whole_2015}.

It is important to obtain the epithermal and thermal spectra to produce $^{233}$U from $^{232}$Th because the radiative capture cross section of thorium decreases monotonically from $10^{-10}$ MeV to $10^{-5}$ MeV. Hardening the spectrum tends to significantly increase resonance absorption in thorium and decrease the absorptions in fissile and construction materials. Thus, a significant amount fissile material will be needed to make the reactor critical. 
\begin{figure}[ht!] % replace 't' with 'b' to force it to 
  \centering
  \includegraphics[width=\textwidth]{spectrum_zones.png} 
  \caption{Neutron flux energy spectrum in different core regions normalized by unit lethargy for the initial and equilibrium fuel salt composition.}
  \label{fig:spectrum_zones}
\end{figure}

\subsection{Neutron flux}
Figure~\ref{fig:radial_flux} shows the radial distribution of fast and thermal neutron flux for both initial and equilibrium composition. The neutron flux has the same shape for both compositions but the equilibrium case has a harder spectrum. A significant spectral shift was observed for the central region of the core (zone I) when for the outer region (zone II) it is negligible for fast but notable for thermal neutrons. This neutron flux radial distribution is in a good agreement with original ORNL report \cite{robertson_conceptual_1971}. Overall, spectrum hardening during \gls{MSBR} operation should be carefully studied for designing the reactivity control system.
\begin{figure}[ht!] % replace 't' with 'b' to force it to 
  \centering
  \includegraphics[width=\textwidth]{radial_flux.png} 
  \caption{Radial neutron flux distribution for initial and equilibrium fuel salt composition.}
  \label{fig:radial_flux}
\end{figure}
\subsection{Power and breeding distribution}
Table~\ref{tab:powgen_fraction} shows the power fraction in each zone for initial and equilibrium fuel composition. Figure~\ref{fig:pow_den} reflects the normalized power distribution of the \gls{MSBR} quarter core which is the same at both states. For both the initial and equilibrium compositions, fission primarily occurs in the center of the core, namely zone I. The spectral shift during reactor operation results in different power fractions at startup and equilibrium, but most of the power is still generated in zone I. 
%%%%%%%%%%%%%%%%%%%%%%%%%%%%%%%%%%%%%%%%
\begin{table}[ht!]
  \centering
  \caption{Power generation fraction in each zone for initial and equilibrium state.}
\begin{tabularx}{\textwidth}{ m | s | s } \hline
Core region      & Initial      & Equilibrium   \\   \hline
Zone I           & 97.91\%      & 98.12\%   \\
Zone II          & 2.09\%       & 1.88\%   \\ \hline
\end{tabularx}
  \label{tab:powgen_fraction}
\end{table}
%%%%%%%%%%%%%%%%%%%%%%%%%%%%%%%%%%%%%%%%%%%%%%%%%%%%%%%%%%%%%%%%%%%%%%%%%%%%%%%%
Figure~\ref{fig:breeding_den} shows the neutron capture reaction rate distribution for $^{232}$Th normalized by the total neutron flux for initial and equilibrium states. The distribution reflects the spatial distribution of $^{233}$Th production in the core. The thorium-232 then $\beta$-decays to $^{233}$Pa which is the precursor for $^{233}$U production. Accordingly, this characteristic represents the breeding distribution in the \gls{MSBR} core. Spectral shift does not cause significant changes in power nor in breeding distribution. Even after 20 years of operation, most of the power still is generated in zone I though the majority of $^{233}$Th is produced in zone II.
\begin{figure}[ht!] % replace 't' with 'b' to force it to 
  \centering
  \includegraphics[width=\textwidth]{power_distribution_eq.png} 
  \caption{Normalized power density for both initial and equilibrium fuel salt composition.}
  \label{fig:pow_den}
\end{figure}
\begin{figure}[ht!] % replace 't' with 'b' to force it to 
  \centering
  \includegraphics[width=\textwidth]{breeding_distribution_eq.png} 
  \caption{$^{232}$Th neutron capture reaction rate normalized by total flux for both initial and equilibrium fuel salt composition.}
  \label{fig:breeding_den}
\end{figure}
\subsection{Temperature coefficient of reactivity}
Table~\ref{tab:tcoef} summarizes temperature effects on reactivity calculated in this work for both initial and equilibrium fuel composition, and compared with original \gls{ORNL} report data \cite{robertson_conceptual_1971}. Uncertainty for each temperature coefficient also appears in Table~\ref{tab:tcoef}. The main physical principle underlying the reactor temperature feedback is an expansion of matterial that is heated. When the fuel salt temperature increases, the density of the salt decreases, but at the same time, the total volume of fuel salt in the core remains constant because it is bounded by the graphite. When the graphite temperature increases, the density of graphite decreases creating additional space for fuel salt. To determine temperature coefficients, the cross section temperatures for fuel and moderator were changed from 900K to 1000K. Three different cases were considered:
\begin{enumerate}
  \item Temperature of fuel salt rising from 900K to 1000K.
  \item Temperature of graphite rising from 900K to 1000K.
  \item Whole reactor temperature rising from 900K to 1000K.
\end{enumerate}
%%%%%%%%%%%%%%%%%%%%%%%%%%%%%%%%%%%%%%%%
\begin{table}[ht!]
  \centering
  \caption{Temperature coefficients of reactivity for initial and equilibrium state.}
\begin{tabularx}{\textwidth}{ s | s | s | x } \hline
   Reactivity coefficient [pcm/K]  & Initial      & Equilibrium  & Reference \cite{robertson_conceptual_1971} \\  \hline
Fuel salt        & $-3.22\pm0.044$ & $-1.53\pm0.046$ & $-3.22$  \\
Moderator        & $+1.61\pm0.044$ & $+0.97\pm0.046$ & $+2.35$  \\
Total            & $-3.1\pm0.04$   & $-0.97\pm0.046$ & $-0.87$  \\ \hline
\end{tabularx}
  \label{tab:tcoef}
\end{table}
%%%%%%%%%%%%%%%%%%%%%%%%%%%%%%%%%%%%%%%%%%%%%%%%%%%%%%%%%%%%%%%%%%%%%%%%%%%%%%%%
In the first case, changes in the fuel temperature only impact fuel density. In this case, the geometry is unchanged because the fuel is a liquid. However, when the moderator heats up, both the density and the geometry change due to thermal expansion of the solid graphite blocks and reflector. Accordingly, the new graphite density was calculated using a linear temperature expansion coefficient of 1.3$\times10^{-6}$1/K \cite{robertson_conceptual_1971}. A new geometry input was created based on this information.

The fuel temperature coefficient (FTC) is negative for both initial and equilibrium fuel compositon due to thermal Doppler broadening of the resonance capture cross sections in the thorium and is in a good agreement with earlier research \cite{robertson_conceptual_1971,park_whole_2015}. The moderator temperature coefficient (MTC) is positive for startup composition and decreases during reactor operation because of spectrum hardening with fuel depletion. Finally, the total temperature coefficient of reactivity is negative for both cases, but decreases during reactor operation due to spectral shift. In summary, even after 20 years of operation the total temperature coefficient of reactivity is relatively large and negative during reactor operation, despite positive MTC, and affords excellent reactor stability and control.

\subsection{Reactivity control system rod worth}
Table~\ref{tab:rod_worth} summarizes the reactivity control system worth. During normal operation the control (graphite) rods are fully inserted, and the safety (B$_4$C) rods are fully withdrawn. To insert negative reactivity into the core, the graphite rods are gradually withdrawn from the core. In an accident, the safety rods would fall down into the core. The integral rod worths were calculated for various positions to separately estimate control (graphite) rod, safety (B$_4$C) rod, and the whole reactivity control system worth. Control rod integral worth is approximately 28 cents and stays almost constant during reactor operation. The safety rod integral worth decreases by  16.2\% during 20 years of operation because of neutron spectrum hardening and absorber accumulation in proximity to reactivity control system rods. This 16\% decline in control system worth should be taken into account in \gls{MSBR} accident analysis and safety justification.
%%%%%%%%%%%%%%%%%%%%%%%%%%%%%%%%%%%%%%%%
\begin{table}[ht!]
  \centering
  \caption{Control system rod worth for initial and equilibrium fuel composition.}
\begin{tabularx}{\textwidth}{ b | x | x } \hline
Reactivity parameter [cents]  &  Initial      &  Equilibrium      \\ \hline
Control (graphite) rod integral worth               & $\ 28.2\pm0.8$    & $\ 29.0\pm0.8$ \\ 
Safety (B$_4$C) rod integral worth                  & $251.8\pm0.8$    & $211.0\pm0.8$  \\
Total reactivity control system worth               & $505.8\pm0.7$    & $424.9\pm0.8$ \\ \hline
\end{tabularx}
  \label{tab:rod_worth}
\end{table}
%%%%%%%%%%%%%%%%%%%%%%%%%%%%%%%%%%%%%%%%%%%%%%%%%%%%%%%%%%%%%%%%%%%%%%%%%%%%%%%%

\subsection{Six Factor Analysis}
The effective multiplication factor could be expressed using formula:
\begin{align*}
k_{eff} = k_{inf} P_f  P_t = \eta \epsilon p f P_f P_t
\end{align*}

Table~\ref{tab:six_factor} summarizes the six factors for both initial and equilibrium fuel salt composition. The non-leakage probability for both fast and thermal neutrons does not change during reactor operation because these values are not largely affected by the neutron spectrum shift. In contrast, neutron reproduction factor ($\eta$), resonance escape probability (p), and fast fission factor ($\epsilon$) are considerably different between startup and equilibrium. As indicated in Figure~\ref{fig:spectrum} the neutron spectrum is softer for the initial state. Neutron spectrum hardening causes the fast fission to increase through the core lifetime. The opposite is true for the resonance escape probability. Finally, the neutron reproduction factor decreases during reactor operation due to accumulation of fissile plutonium isotopes.
%%%%%%%%%%%%%%%%%%%%%%%%%%%%%%%%%%%%%%%%
\begin{table}[hb!]
  \centering
  \caption{Six factors for the full-core \gls{MSBR} model for initial and equilibrium fuel composition.}
\begin{tabularx}{\textwidth}{ b | s | s } \hline
Factor  & Initial      & Equilibrium   \\ \hline
Neutron reproduction factor ($\eta$)     & $1.3960\pm.000052$     & $1.3778\pm.00005$ \\ 
Thermal utilization factor (f)           & $0.9670\pm.000011$     & $0.9706\pm.00001$ \\
Resonance escape probability (p)         & $0.6044\pm.000039$     & $0.5761\pm.00004$ \\
Fast fission factor ($\epsilon$)         & $1.3421\pm.000040$     & $1.3609\pm.00004$ \\
Fast non-leakage probability (P$_f$)     & $0.9999\pm.000004$     & $0.9999\pm.000004$ \\
Thermal non-leakage probability (P$_t$)  & $0.9894\pm.000005$     & $0.9912\pm.00005$ \\ \hline
\end{tabularx}
  \label{tab:six_factor}
\end{table}
%%%%%%%%%%%%%%%%%%%%%%%%%%%%%%%%%%%%%%%%%%%%%%%%%%%%%%%%%%%%%%%%%%%%%%%%%%%%%%%%
\subsection{Thorium refill rate}
In \gls{MSBR} reprocessing scheme the only external feed material flow  is $^{232}$Th. Figure~\ref{fig:th_refill} shows the $^{232}$Th feed rate calculated for 60 years of reactor operation. The the $^{232}$Th feed rate fluctuates significantly as a result of the batch-wise nature of this online reprocessing approach. For example, the large spikes up to 36 kg/day in a thorium consumption occurs every 3435 days. This is required due to batch-wise removal of strong absorbers (Rb, Sr, Cs, Ba). The corresponding effective multiplication factor increase (Figure~\ref{fig:keff}) and breeding intensification leads to additional $^{232}$Th consumption. 
\begin{figure}[ht!] % replace 't' with 'b' to force it to 
  \centering
  \includegraphics[width=\textwidth]{Th_refill_rate.png} 
  \caption{$^{232}$Th feed rate over 60 years of \gls{MSBR} operation.}
  \label{fig:th_refill}
\end{figure}

The average thorium feed rate increases during the first 500 days of operation and than steadily decreases due to spectrum hardening and accumulation of absorbers in the core. As a result, the average $^{232}$Th feed rate over 60 years of operation is about 2.40 kg/day. This results are in a good agreement with a recent online reprocessing study by \gls{ORNL} \cite{betzler_molten_2017}.

\subsection{The effect of removing fission product from fuel salt}
Loading initial fuel salt composition into the \gls{MSBR} core leads to supercritical configuration (Figure ~\ref{fig:fp_removal}). After reactor startup the effective multiplication factor for the case with volatile gases and noble metals removal is approximately 7500 pcm  higher than for case with no fission products removal. This significant impact on the reactor core achieved due to immediate removal (20 sec cycle time) and high absorption cross section of Xe, Kr, Mo, and other noble metals removed. The effect of rare earth element removal considerable after few month from startup and achieves approximately 5500 pcm after 10 years of operation. The rare earth elements are removed with slower rate (50 days cycle time). Moreover, Figure~\ref{fig:fp_removal} demonstrates that batch-wise removal every 3-day step even strong absorbers did not necessarily leads to fluctuation in results but rare earth elements removal every 50 days causes approximately 600 pcm jump in reactivity.

The effective multiplication factor of the core reduces gradually over operation time because the fissile material ($^{233}$U) continuously depletes from the fuel salt due to fission and, at the same time, fission products accumulates in the fuel salt.	Eventually, without fission products removal, the reactivity decreases to the subcritical state after approximately 500 and 1300 days of operation for cases with no removal and volatile gases \& noble metals removal, respectively. The time when the simulated core reaches subcriticality ($k_{eff}<$1.0) for full-core model) called the core lifetime. Therefore, removing fission products provides with significant neutronic benefit because allows achieve longer core lifetime.
\begin{figure}[ht!] % replace 't' with 'b' to force it to 
  \centering
  \includegraphics[width=\textwidth]{keff_rem_cases.png} 
  \caption{Calculated effective multiplication factor for full-core \gls{MSBR} model with removal of different fission product groups over 10 years of operation. The confidence interval $\pm\sigma	$ is shaded.}
  \label{fig:fp_removal}
\end{figure}